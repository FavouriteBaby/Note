\documentclass{article}
\usepackage{ctex}
\usepackage{graphics}
\usepackage{geometry}
\usepackage{listings}
\usepackage[colorlinks,linkcolor=red,anchorcolor=blue,citecolor=green]{hyperref}
% bmeps -c test.png test.eps
\geometry{left=2.0cm,right=2.0cm,top=2.0cm,bottom=2.0cm}
\title{正则表达式}
\author{pwlin1992@gmail.com}
\begin{document}
\maketitle
\tableofcontents
\begin{large}
  \section{语法}
  \subsection{+*?}
  \begin{enumerate}
  \item +

    +号代表前面的字符必须至少出现1次。
  \item *

    *号代表字符可以不出现,也可以出现1次或多次。
  \item ?

    ?号代表前面的字符最多只能出现1次(0次,1次)。

    如:colou?r,表示color或colour
  \end{enumerate}

  \subsection{普通字符与特殊字符}
  正则表达式是由普通字符及特殊字符组成的文字模式。

  \paragraph{普通字符}\mbox{} \\

  大写、小写字母,所有数字,所有标点符号和一些其他符号。

  非打印字符:

  \begin{tabular}{|c|c|}
    \hline
    字符 & 描述 \\
    \hline
    \verb|\cx| & 匹配由x指明的控制字符。如\verb|\cM|匹配回车。\\
    \hline
    \verb|\f| & 匹配换页符,等价于\verb|\cL| \\
    \hline
    \verb|\n| & 匹配一个换行符,等价于\verb|\cJ| \\
    \hline
    \verb|\r| & 匹配一个回车符。 \\
    \hline
    \verb|\s| & 匹配任何空白字符,如空格,制表符,换页符等。 \\
    \hline
    \verb|\S| & 匹配任何非空白字符。 \\
    \hline
    \verb|\t| & 匹配一个制表符。 \\
    \hline
    \verb|\v| & 匹配一个垂直制表符。\\
    \hline
  \end{tabular}

  \paragraph{特殊字符}\mbox{} \\
  指的是正则表达式中有特殊意义的字符,如\verb|*$+.{|等。要匹配这些字符,都要用\verb|\|来转义。
    \begin{tabular}{|c|c|}
      \hline
      字符 & 描述 \\
      \hline
      \verb|$| & 匹配输入字符串的结尾位置。\\
      \hline
      \verb|()| & 标记一个子表达式的开始和结束位置。\\
      \hline
      \verb|*| & 匹配前面的子表达式零次或多次。\\
      \hline
      \verb|+| & 匹配前面的子表达式一次或多次。 \\
      \hline
      \verb|.| & 匹配除换行符之外的任何单字符。 \\
      \hline
      \verb|[| & 标记一个中括号表达式的开始。 \\
      \hline
      \verb|?| & 匹配前面的子表达式零次或一次。 \\
      \hline
      \verb|\| & 标记下一个字符为特殊字符、或原义字符、或向后引用、或八进制转义符。 \\
      \hline
      \verb|^| & 匹配输入字符串的开始位置。 \\
      \hline
      \verb|{| & 标记限定符表达式的开始。 \\
      \hline
      \verb'|' & 指明两项之间的一个选择。 \\
      \hline
    \end{tabular}

    \subsection{限定符}
    指出正则表达式的一个给定组件必须要出现多少次才能满足匹配。有*,+,?,\verb|{n},{n,},{n,m}|6种。
    \begin{tabular}{|c|c|}
      \hline
      字符 & 描述 \\
      \hline
      \verb|*| & 等价于{0,} \\
      \hline
      \verb|+| & 等价于{1,} \\
      \hline
      \verb|?| & 等价于{0,1} \\
      \hline
      {n} & 匹配确定的n次,如o{2}可以匹配food中的o,但不能匹配Bob中的o \\
      \hline
      {n,} & 至少匹配n次。\\
      \hline
      {n,m} & 最少匹配n次,最多匹配m次。\\
      \hline
    \end{tabular}

    \subsection{定位符}
    定位符使人可以将正则表达式固定到行首或行尾,或是单词内、单词开头、单词结尾。
    \begin{tabular}{|c|c|}
      \hline
      字符 & 描述 \\
      \verb|^| & 匹配输入字符串开始的位置。 \\
      \hline
      \verb|$| & 匹配输入字符串结尾的位置。\\
      \hline
      \verb|\b| & 匹配一个字边界,即字与空格间的位置。\\
      \hline
      \verb|\B| & 非字边界匹配。\\
      \hline
    \end{tabular}
    如果\verb|\b|位于要匹配的字符串的开始处,则在单词的开始处查找匹配项,如果位于字符串的结尾,则在单词的结尾处查找匹配项。
\end{large}
\end{document}
