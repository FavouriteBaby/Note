\documentclass{article}
\usepackage{ctex}
\usepackage{graphics}
\usepackage{geometry}
\usepackage{listings}
\usepackage[colorlinks,linkcolor=red,anchorcolor=blue,citecolor=green]{hyperref}
% bmeps -c test.png test.eps
\geometry{left=2.0cm,right=2.0cm,top=2.0cm,bottom=2.0cm}
%\definecolor{cGray}{RGB}{230,230,250}
\newcommand{\code}[1]{\colorbox[RGB]{255,182,193}{\textcolor[RGB]{220,20,60}{#1}}}
\title{Cyber Security Situational Evaluation Model Based On Big Data And Perceptual Hash}
\author{pwlin1992@gmail.com}
\begin{document}
\maketitle
\tableofcontents
\begin{large}

\section{Abstract}

\begin{enumerate}
\item 阐述问题
\item 说明自己的解决方案和结果
\end{enumerate}
\section{Introduction}

\begin{enumerate}
\item 网络安全态势感知的重要性
\item 国内外研究现状,介绍A、B、C曾研究过这个问题
\item ABC中存在的一些缺陷
\item 自己提出的方法D
\item D的基本特征,与ABC进行比较
\item 实验证明比ABC更好
\item 论文的结构
\end{enumerate}

\section{Related Work}

重点在于说明自己与前人的不同之处
\begin{enumerate}
\item 将前人的工作分门别类
\item 对每项重要的历史工作进行简要的概述
\item 和自己提出的工作进行比较
\item 不能忽略前人的重要工作,公正评价前人的工作
\item 强调自己的工作和前人工作的不同,举出各自适用的场景
\end{enumerate}

\section{Our Solution}

重点在于描述自己的工作。
\begin{enumerate}
\item 以读者的角度,给出相关的定义及表示法
\item 提供算法的伪代码,图解,相应的解释
\item 以读者的角度思考读者可能提出的问题,在论文中给出回答
\end{enumerate}

\section{Performance Analysis}

验证提出的方法和思路
\begin{enumerate}
\item 设计实验:简洁的实验和详尽的实验步骤
\item 必要的比较,突出科学性
\item 讨论,说明结果的意义
\item 得出结论
\end{enumerate}

\subsection{节点安全态势对比}

\subsection{局部网络安全态势对比}

\subsection{网络安全态势对比}

\subsection{效率}

\subsection{实施性}

\subsection{存储空间}

\subsection{复杂度}

\section{Conclusion}

总结,前景及结尾
\begin{enumerate}
\item 快速简短的总结
\item 未来工作的展望
\item 结束全文
\end{enumerate}

\section{References}

\end{large}
\end{document}
%%% Local Variables:
%%% mode: latex
%%% TeX-master: t
%%% End:
